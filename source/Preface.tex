%!TEX root = Funktionalanalysis - Vorlesung.tex

%\chapter*{Vorwort}
%Dieses Skript wurde im Wintersemester 2015/2016
%von Martin Belica geschrieben. Es beinhaltet die Mitschriften aus
%der Vorlesung von Prof.~Dr.~Szech sowie die Mitschriften einiger
%Übungen.


\chapter*{Content of teaching}

The course covers topics from behavioral economics with regard to contents and methods. In addition, the students gain insight into the design of economic experiments. Furthermore, the students will become acquainted with reading and critically evaluating current research papers in the field of behavioral economics. \\
  
\textbf{Prerequisites} \\
None. Recommendations: Basic knowledge of microeconomics and statistics are recommended. A background in game theory is helpful, but not absolutely necessary. \\

\textbf{Aim} \\
The students gain insight into fundamental topics in behavioral economics;
get to know different research methods in the field of behavioral economics;
learn to critically evaluate experimental designs;
get introduced to current research papers in behavioral economics;
become acquainted with the technical terminology in English. \\

\textbf{Bibliography} \\
\begin{itemize}
	\item Kahnemann, Daniel: Thinking, Fast and Slow. Farrar, Straus and Giroux, 2011.
	\item Ariely, Dan: Predictably irrational. New York: Harper Collins, 2008.
	\item Ariely, Dan: The Upside of Irrationality. New York: HarperCollins, 2011. 
\end{itemize}	
