%!TEX root = Funktionalanalysis - Vorlesung.tex


\chapter*{Preface}
This script has been created by Martin Belica in WS 2015/16. It is an \textbf{unofficial} script and contains notes from the lectures by Prof.~Dr.~Szech at the KIT as well as from some exercises.
\section*{Content of teaching}
The course covers topics from behavioural economics with regard to contents and methods. In addition, the students gain insight into the design of economic experiments. Furthermore, the students will become acquainted with reading and critically evaluating current research papers in the field of behavioural economics. \\

\textbf{Prerequisites} \\
None. Recommendations: Basic knowledge of microeconomics and statistics are recommended. \\

\textbf{Aim} \\
The students gain insight into fundamental topics in behavioural economics;
get to know different research methods in the field of behavioural economics;
learn to critically evaluate experimental designs;
get introduced to current research papers in behavioural economics;
become acquainted with the technical terminology in English. \\

\textbf{Bibliography}
\begin{itemize}
	\item Kahnemann, Daniel: Thinking, Fast and Slow. Farrar, Straus and Giroux, 2011.
	\item Ariely, Dan: Predictably irrational. New York: Harper Collins, 2008.
	\item Ariely, Dan: The Upside of Irrationality. New York: HarperCollins, 2011. 
\end{itemize}
~\newline
\textbf{Exam information} (unofficial) \\
The exam will...
\begin{itemize}
	\item last 1h with 60 points to achieve, which means in average 1 minute per point is planned
	\item consist of 2-4 exercises
	\item have 2-4 subtasks per exercise 
\end{itemize}
Theory is relevant but the papers are the focus: one has to be able to explain the design, recap main questions and results and maybe to argue about importance, errors and improvement suggestions.
