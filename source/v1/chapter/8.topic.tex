%!TEX root = Economics and Behaviour.tex


\chapter{Ethics in science}

\section{Pleasures of Skill and Moral Conduct}

Background:
\begin{itemize}
	\item Jeremy Bentham pointed fourteen different $"$simple$"$ sources of pleasures for humans out
	\item In this short list, number three is the $"$pleasure of skill$"$ while number five is $"$the pleasure of a good name$"$.
	\item Yet if being skilful is of crucial importance to people than this can oppose the possibility to keep a good name
\end{itemize}
  
  
  
\textbf{As an example: The Manhattan Project.} After the dropping of the plutonium bomb on Nagasaki, numerous members of the Manhattan Project started worrying about moral implications. Many of the scientists suffered from e.g depressions. \\ 

The Self-Image is so relevant in this concept. Both the desire for mastery and acting in accordance with moral values originate from the same source, a desire for positive self-image. \\ 

The remaining question is therefore: does morality in some (everyday) situations get traded off against skilfulness?


\section{Moral and Markets}

Examples for market designs where the idea of introducing a free market (money based) is current:
\begin{itemize}
	\item trading markets for emission certificates. To reduce pollution by restricting emission output per country a contract was design, which nevertheless allowed trading of those certificates. M. Sandel was concerned that if we put a money value on pollution it might become less moral concerning to pollute.
	\item Allocation of organs markets: one might be able to trade an incompatible organ for an compatible if available. People started discussing if money should not be introduced in this market instead of just a trading market.
	\item Adoption: high income families might be able to provide better for adopted children and therefore could be preferred on an adoption list
	\item In California child baring is allowed to be traded for money 
\end{itemize}

Restricted markets:

\begin{itemize}
	\item Employment markets are regulated, so exploitation is not (so) present.
\end{itemize}



Next, the paper \textit{moral and markets} by Prof. Szech was discussed, issuing the topic:


The possibility that market interactions may erode moral values is a long-standing, but controversial, hypothesis in the social sciences, ethic and philosophy. Markets are accused to transform human values in exchange blues and goods into commodities. It has also been argues that market institutions may influence preferences in general with a tendency to make people.

Michael Sandal analysed that with technological progress and the increasing ubiquity of market ideas, since markets continue to enter further and further domains of our social life.


Further, there is the doux commerce hypothesis, meaning that the entering of market in our social life might improve our situation in many ways...


\section{Presented papers}

\begin{itemize}
	\item Falk, A.; Szech, N. (2016): \textit{Pleasures of Skill and Moral Conduct}. KIT working paper. (non-monetary incentives and morals)
		\begin{itemize}
			\item As was recognised by Bentham, skilfulness is an important source of pleasure. Humans like achievement and to excel in tasks relevant to them. This paper provides controlled experimental evidence that striving for pleasures of skill can have negative moral consequences and causally reduce moral values. In the study, subjects perform an IQ-test. They know that each correctly solved question not only increases test performance but also the likelihood of moral transgression. In terms of self-image, this creates a trade-off between signalling excellence and immoral disposition. We contrast performance in the IQ-test to test scores in an otherwise identical test, which is, however, framed as a simple questionnaire with arguably lower self-relevance. We find that subjects perform significantly better in the IQ-test condition, and thus become more willing to support morally problematic consequences. Willingness to reduce test performance in order to behave more morally is significantly less pronounced in the IQ versus the more neutral context. The findings provide controlled and causal evidence that the desire to succeed in a challenging, self-relevant task has the potential to seduce subjects into immoral behaviours and to significantly decrease values attached to moral outcomes.
		\end{itemize}
	\item Russell, B. (1960): \textit{The Social Responsibilities of Scientists}. In: Science, New Series.
		\begin{itemize}
			\item A scientist can no longer shirk responsibility for the use society makes of his discoveries.
		\end{itemize}
	\item William 0. Baker and more (1961): \textit{The Moral UnNeutrality of Science: The scientist's special responsibility are examined an address given at the 1960 AAAS annual meeting}. In: Science.
		\begin{itemize}
			\item The scientist's special responsibilities are examined in an address given at the 1960 AAAS annual meeting
		\end{itemize}
\end{itemize}


\newpage