%!TEX root = Economics and Behaviour.tex


\chapter{Standard theoretic basics for analysis of strategic behaviour}

A \begriff{game} is a formal representation of a situation in which a number of individuals interact in a setting of \textit{strategic interdependence}.
To be fully defined a game must specify the following set of elements $\{ N, S, u \}$ where
	\begin{enumerate}
		\item $N$ is the finite number of players and for player $n$ would that mean $n \in \{ 1, \dotsc, N \}$.
		\item For each player we have a set of \textit{pure} strategies $S$.
		\item For each player $n \in \{1, \dotsc, N \}$ we have an expected utility function $u : S \rightarrow \MdR$
	\end{enumerate}

For example Tick-Tack-Toe games, auctions or even meetings can be described with games. 

In the following two sections equilibria are one of our main focus point. Three properties describe such a state, where economic forces are balanced and in the absence of external influences the behaviour leading to the equilibrium will not change:
\begin{enumerate}
	\item The behaviour of agents in such a point is consistent.
	\item No agent has an incentive to change its behaviour.
	\item The equilibrium is the outcome of some dynamic process (stability).
\end{enumerate}