%!TEX root = Economics and Behaviour.tex


\chapter{Standard theoretic basics for analysis of strategic behaviour}

First, a strategic interaction occurs when the utility of agents in a situation is mutually influenced by individual behavioural changes. 

A \begriff{game} is a formal representation of a situation in which a number of individuals interact in a setting of strategic interdependence.
	It is necessary to clarify four points four a game:
	\begin{itemize}
		\item The players: Who is interacting?
		\item The rules: When do they move or what can they do?
		\item The outcomes: For each possible set of actions by the players, what is the outcome of the game?
		\item The payoffs: What are the players' preferences over the possible outcomes?
	\end{itemize}

For example Tick-Tack-Toe games, auctions or even meetings can be described with games. 

\index{equilibrium}
In the following two sections equilibria are one of our main focus point. Three properties describe such a state, where economic forces are balanced and in the absence of external influences the behaviour leading to the equilibrium will not change:
\begin{enumerate}
	\item The behaviour of agents in such a point is consistent.
	\item No agent has an incentive to change its behaviour.
	\item The equilibrium is the outcome of some dynamic process (stability).
\end{enumerate}

~\newline
Subsequent, we distinguish between the formal representation which we use to model the conflict situation. However, all situations we will examine will have complete information, meaning perfect knowledge (esp. structure of the game, possible actions and payoff functions of all players) is available to all individuals. 


\newpage