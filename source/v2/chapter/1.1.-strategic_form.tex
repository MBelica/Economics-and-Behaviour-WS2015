%!TEX root = Economics and Behaviour.tex

\section{Strategic form}

In this first section we will merely consider \begriff{static games} where all players choose their individual actions simultaneously (see One-Shot games). Yet, the term simultaneously is meant figuratively. It is not important that all players act at the same time but that at the time of their choice the decisions of all other players are unknown\footnote{Experiments have shown that even if theoretically the decision process should be the same people tend to behave differently if they have knowledge about a decision order  (see Rapoport 1997, Guth, Huck und Rapoport 1998)}.

In static games of complete, perfect information, a strategic form or normal-form representation of a game is a specification of players' strategy spaces and payoff functions. \\

First let's take a look at the definition of a strategy, which we will then use to define the strategic form of games: 

\begin{definition}[Strategy]
	Let $\mathcal{H}_{i}$ denote the collection of player $i$'s sets of information, $\mathcal{A}$ the set of possible actions in the game and $C(H) \subseteq \mathcal{A}$ the set of actions possible at information set $H$. A \begriff{strategy} for player $i$ is a function $s_{i} \colon \mathcal{H}_{i} \rightarrow \mathcal{A}$ such that
	\[ s_{i}(H) \in C(H) \text{ for all } H \in \mathcal{H}_{i} \]
\end{definition}

We call a set of strategies a \begriff{complete plan} of actions for each situation in a game and with $S_{-i}$ we denote the strategies of all players except player $i$.\\

\begin{definition}[Strategic form representation]
To be fully defined a game in \begriff{strategic form} must specify the set $\{ N, S, u \}$ where
	\begin{enumerate}
		\item $N$ is the finite number of players and for player $i$ that would mean $i \in \{ 1, \dotsc, N \}$.
		\item For each player $i$ we have a set of strategies $S_{i}$, such that $S = \bigotimes S_{i}$.
		\item For each player $i$ we have an expected utility function $u_{i} : S \rightarrow \MdR$, such that $u = \{ u_{1}, \dotsc, u_{N} \}$.
	\end{enumerate}
\end{definition}



To visualise a static game with two players ($P1$ and $P2$) and a finite number of possible strategies (for simplicity let's assume that there are only two signals and call them $a$ and $b$) one commonly uses the \begriff{matrix form}, where $u_{i}(x, y)$ represents the utility function for player $i$ given the strategy $x$ for $P1$ and $y$ for $P2$ with $x, y \in \{ a, b\}$.
\begin{center}
	\begin{tabular}{|c|c|c|}
		\hline\hline
  			$P1$ / $P2$ & \textbf{a} & \textbf{b} \\
         		\cline{1-3}
   					\textbf{a} & $( u_{1}(a, a) , u_{2}(a, a))$ & $(u_{1}(a, b), u_{2}(a, b))$	\arrayrulewidth2pt \\
            	\cline{1-3}
   					\textbf{b} & $( u_{1}(b, a), u_{2}(b, a))$ & $(u_{1}(b, b), u_{2}(b, b))$\\ \hline\hline
	\end{tabular}	
\end{center}

\begin{example}[Prisoner's Dilemma] \label{prisonersdilemma} \index{Prisoner's Dilemma}
	 Imagine, two members of a criminal gang are arrested and imprisoned. Each prisoner is in solitary confinement with no means of communicating with the other. The prosecutors lack sufficient evidence to convict the pair on the principal charge. They hope to get both sentenced to a year in prison on a lesser charge. Simultaneously, the prosecutors offer each prisoner a bargain. Each prisoner is given the opportunity either to: betray the other by testifying that the other committed the crime, or to cooperate with the other by remaining silent. The offer is:
	\begin{itemize}
		\item If A and B betray each other, each of them serves 6 years in prison
		\item If A betrays B but B remains silent, A will be set free and B will serve 9 years in prison (and vice versa)
		\item If A and B both remain silent, both of them will only serve 1 year in prison (on the lesser charge)
	\end{itemize}
	
	\begin{center}
		\begin{tabular}{|l|l|r|}
			\hline\hline
  				P1 / P2 & \textbf{defects} & \textbf{cooperates} \\
         			\cline{1-3}
   				\textbf{defects} & $(-6, -6)$ & $(0, -9)$ 	\arrayrulewidth2pt \\
            		\cline{1-3}
   				\textbf{cooperates} & $(-9, 0)$ & $(-1, -1)$ \\
			\hline\hline
		\end{tabular}	
	\end{center}
\end{example}

There are many different possibilities to extrapolate the Prisoner's Dilemma to apply it in a variety of problems, other interpretations are for example:
\begin{itemize}
	\item Collusion on prices
	\item Investing in human capital vs. arming for a war
	\item Buying a SUV vs. a smaller car
\end{itemize}

Now that we got to know an example for a game we should discuss some solution concepts and the first obvious choice is the strict dominance.

\begin{definition}[Strict dominance]
	A strategy $s_{i} \in S_{i}$ is \begriff{strictly dominant} for player $i$ if for all $s_{i}' \neq s_{i} (s_{i}'  \in S_{i})$: 
	\[ u(s_{i}, s_{-i}) > u(s_{i}', s_{-i}), \quad \forall s_{-i} \in S_{-i} \]	
\end{definition}

Analysing \hyperref[prisonersdilemma]{Prisoner's Dilemma} one can see that $cooperate$ is strictly dominated by $defect$. Simply the elimination of strictly dominated strategies leads to the prediction that the players choose $(defects, defects)$ even though $(cooperates, cooperates)$ would result in a lower prison sentence. 

This leads us to the elimination of irrational strategies:

\begin{definition}[Best response]
	The strategy $s_{i}$ is a \begriff{best response} for player $i$ to the opponent's strategies $s_{-i}$ if
	\[ u_{i}(s_{i}, s_{-i}) \geq u_{i}(s_{i}', s_{-i}) \text{ for all } s_{i}' \in S_{i} \]
	A strategy $s_{i}$ is never a best response if there is no $s_{-i}$ for which $s_{i}$ is a best response.
\end{definition}

\begin{definition}[Rationalisable Strategies]
	The strategies that survive the iterated elimination of strategies that are never a best response are known as player $i$'s \index{rationalisable strategies}.
\end{definition}

Iteratively eliminating dominated strategies leads to a set of rationalisable strategies; for example:

	\begin{center}
		\begin{tabular}{|r|r|r|r|}
			\hline\hline
  				P1 / P2 & \textbf{l} & \textbf{m} & \textbf{r} \\
         			\cline{1-4}
   				\textbf{u} & $(1, 1)$ & $(2, 2)$ & $(2, 0)$ \arrayrulewidth2pt \\
            		\cline{1-4}
   				\textbf{m} & $(2, 0)$ & $(0, 1)$ & $(1, 0)$ \arrayrulewidth2pt \\
            		\cline{1-4}
   				\textbf{d} & $(0, 2)$ & $(1, 1)$ & $(1, 1)$ \\			\hline\hline
		\end{tabular}	
	\end{center}
	
Here, an iterated elimination leads to $(u, m)$ as rationalisable strategies:
\begin{itemize}
	\item For Player 1 $d$ is strictly dominated by $u$ and should therefore never be played.
	\item For Player 2 $r$ is strictly dominated by $m$.
	\item Since Player 1 would never play $d$, $m$ dominates in the iterative subgame $l$
	\item Now knowing Player 2 should play $m$, $u$ is the rational choice for Player 1
\end{itemize} 	

Important to notice is that here, the predictions we derived rely immensely on the rationality of all players.

Since there is not always a strictly dominant strategy we extend our solution concepts with the Nash-Equilibrium.

\begin{definition}[Nash-Equilibrium] \label{nashequilibrium} 
A strategy set $s = (s_{1}, \dotsc, s_{N})$ constitutes a \begriff{Nash-Equilibrium} of a game if for every $i = 1, \dotsc, N$ (where $N$ is the number of players)
	\[ u_{i}(s_{i}, s_{-i}) \geq u_{i}(s_{i}', s_{-i}) \text{ for all } s_{i}' \in S_{i} \]
\end{definition}

In other words, a \begriff{Nash-Equilibrium} is the mutual best response for every player, therefore a set of strategies in which no player can do better by unilaterally changing their strategy. \\

\begin{example}[Battle of the sexes] \label{battleofthesexes} \index{Battle of sexes}
		Imagine a couple that agreed to meet this evening, but both individually cannot recall if they will be attending the opera or a football match. The husband would most of all like to go to the football game. The wife would like to go to the opera. Both would prefer to go to the same place rather than different ones. \\ \\
		Hence, the Battle of the sexes in strategic form could, of course depending on their utility function, look something like:
		\begin{center}
			\begin{tabular}{|l|l|r|}
				\hline\hline
  					M / F & \textbf{football} & \textbf{opera} \\
         				\cline{1-3}
   					\textbf{football} & $(1, 2)$ & $(0, 0)$ 	\arrayrulewidth2pt \\
            			\cline{1-3}
   					\textbf{opera} & $(0, 0)$ & $(2, 1)$ \\ \hline\hline
			\end{tabular}	
		\end{center}
		
		and the two Nash-Equilibriums in this game are $(opera, opera)$ and $(football, football)$.
\end{example}

\begin{example}[The Beauty-Contest] \index{Beauty-Contest} \label{Beauty-Contest}
John Keynes described the action of rational agents in a market using an analogy based on a fictional newspaper contest, in which entrants are asked to choose the six most attractive faces from a hundred photographs. Those who picked the most popular faces are then eligible for a prize. An agent has to consider that not his preferred choice is the optimal strategy but the one with the highest chances to be chosen by all others.
\end{example}
We can generalise this example to the following \\
\begin{example}[Guessing-Game] \index{Guessing-Game} \label{Guessing-Game}
	 A Guessing-Game (e.g. the \hyperref[Beauty-Contest]{Beauty-Contest}) is a game with at least two players in which the sequel can be described as follows:
	\begin{itemize}
		\item Every player guesses a number $b_{i} \in \{0, 1, 2, \dotsc, 100 \}$
		\item The player with the closest guess to $p \cdot \sum_{i = 1}^{n} \frac{b_{i}}{n} = p \cdot \varnothing$ with $p \in (0, 1)$ wins
		\item In case of tie a random device that is 'fair' decides who wins the prize $P > 0$
	\end{itemize}
	\textbf{1. Question:} Is $(0, \dotsc, 0)$ a Nash-Equilibrium? \\
	\textbf{Answer:} Yes. Assume player $i$ bids $b > 0$ and all others bid $0$.	
		\begin{itemize}
			\item if bidder $i$ bids $0$ expected win equals $\frac{1}{n} P $
			\item we can rewrite $p$ times the mean with	
				\[ p \cdot \varnothing = p \cdot \frac{(n - 1)0 + 1 b}{n} = p \cdot \frac{b}{n}. \]
			Then, his expected profit is $0$, as $0$ is closer to $p \cdot \varnothing$ then the bet $b > 0$, since
			\[ \left| b - p \cdot \frac{b}{n} \right| =  \left|(n-p) \cdot \frac{b}{n} \right| \overset{\substack{p < 1, \\ n \geq 2}}{>} \left| p \cdot \frac{b}{n} \right| = \left| 0 - p \cdot \frac{b}{n} \right| \]
		\end{itemize}
	
	\textbf{2. Question:} Is $(0, \dotsc, 0)$ the unique Nash-Equilibrium here? \\
	\textbf{Answer:} Yes, and to bring this to proof we take a look at player $i$'s optimal strategy $b_{i}^{*}$ given the optimal responses of all others:
	
	First, as a consequence our proof of the 1. Question the optimal strategy $b_{i}^{*}$ has to be smaller or equal to the others' winning number: $b_{i}^{*} \leq p \cdot \frac{\sum_{j \neq i} b_{j}^{*}}{n - 1}$.
	
	Considering all players, in terms of summing up over all $i$, yields then
	\[ \sum_{i = 1}^{n} b_{i}^{*} \leq p \cdot \frac{\sum_{i = 1}^{n} \sum_{j \neq i} b_{j}^{*}}{n - 1} = p \cdot \frac{(n - 1) \sum_{j = 1}^{n} b_{j}^{*}}{n - 1} = p \cdot \sum_{j = 1}^{n} b_{j}^{*}. \]
	Thus
		\[ \sum_{i = 1}^{n} b_{i}^{*} \leq p \cdot \sum_{i = 1}^{n} b_{i}^{*}, \quad p \in (0, 1) \]
	has to hold true and with that $b_{i}^{*} = 0$ for all $i$.
	
	Now that a Nash-Equilibrium consists of the mutual best responses only $(0, \dotsc,  0)$ can be a Nash-Equilibrium in this situation. \\

	\textbf{3. Question:} Is $(0, \dotsc, 0)$ also a strictly dominant strategy? \\
	\textbf{Answer:} No. Imagine following situation: \\
	If $48$ of $50$ players bid the number $100$ and the $49$th bids $0$ then the best response for player $50$ is to bid $97$ \footnote{$97$ is the closest whole number to $p \cdot \varnothing$}, which is much larger than 0, so $0$ is not the best answer and therefore cannot be a strictly dominant strategy.
\end{example}


\newpage