%!TEX root = Economics and Behaviour.tex

\section{Level k as a prominent example of a nonstandard/behavioural approach}


\textbf{Unraveling in Guessing Games: An Experimental Study}

\begin{itemize}
	\item Nagel, R. (1995): \textit{Unraveling in Guessing Games: An Experimental Study}. In: American Economic Review.
		\begin{itemize}
			\item Consider the following game: a large number of players have to state in several rounds simultaneously a number in the closed interval [0, 100]. The winner is the person whose chosen number is closest to the mean of all chosen numbers multiplied by a parameter $p$, where $p$ is common knowledge. The payoff to the winner is a fixed amount, which is independent of the stated number and $p$. If there is a tie, the prize is divided equally among the winners. The other players whose chosen numbers are further away receive nothing.
		\end{itemize}
\end{itemize}	
			
\textbf{More than Meets the Eye}

\begin{itemize}
	\item Müller, J.; Schwieren, C. (2011): \textit{More than Meets the Eye: an Eye-tracking Experiment on the Beauty Contest Game}
		\begin{itemize}
			\item The beauty contest game has been used to analyse how many steps of reasoning subjects are able to perform. A common finding is that a majority seem to have low levels of reasoning. We use eye-tracking to investigate not only the number chosen in the game, but also the strategies in use and the numbers contemplated. We can show that not all cases that are seemingly level-1 or level-2 thinking indeed are – they might be highly sophisticated adaptations to beliefs about other people’s limited reasoning abilities.
		\end{itemize}
\end{itemize}		

\newpage