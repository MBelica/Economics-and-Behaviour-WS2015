\chapter*{Introduction} \addcontentsline{toc}{chapter}{Introduction}


In standard economic theory an agent is assumed to be a \begriff{homo oeconomicus}, which portrays the person as consistently rational and narrowly self-interested, usually pursuing his subjectively-defined end optimally; he therefore follows two main characteristics:
\begin{itemize}
	\item \begriff{Rationality}
	\item \begriff{Utility maximisation}
\end{itemize}
However, human behaviour often stands in contrast to this theoretical concept. Behavioural economics examines actual economic behaviour, including widespread cognitive biases and other irrationalities. Such deviations from standard predictions are elicited by various psychological, social, cognitive and emotional factors, and are commonly observed in economical experiments, in which economists collect data to estimate effect size, test the validity of economic theories and illuminate market mechanisms. \\
For these experiments one generally adheres to the following methodological guidelines: \index{economical experiments}
	\begin{itemize}
		\item Incentivise subjects with real monetary payoffs (trustworthiness).
		\item Publish full experimental instructions (transparency).
		\item Do not use deception (honesty).
		\item Avoid introducing specific, concrete context (generalisation).
	\end{itemize}
~	
	
Throughout this notes, we have to keep in mind that every result can be interpreted in various reasonable ways as game theory as a whole can be construed as a predictive tool for the behaviour of human beings, but also as just a suggestion for how people 'ought' to behave. Therefore we subsequently will distinguish between
\begin{itemize}
	\item \begriff{prescriptive} - means containing an indication of approval or disapproval
	\item \begriff{normative} - means relating to a given model
\end{itemize}
approaches.