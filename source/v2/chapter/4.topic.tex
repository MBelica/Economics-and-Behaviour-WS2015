%!TEX root = Economics and Behaviour.tex

\chapter{How strategic interdependence can change subjective decisions}

\section{An ultimatum game with multidimensional response strategies} 

\begin{itemize}
	\item Güth, W.; Levati, M.V.; Nardi, C.; Soraperra, I. (2014): \textit{An ultimatum game with multidimensional response strategies}. In Jena Economic Research Papers, FriedrichSchiller University and Max Planck Institute of Economics, Jena, Germany (Ultimatum Game)
		\begin{itemize}
			\item We enrich the choice task of responders in ultimatum games by allowing them to independently decide whether to collect what is offered to them and whether to destroy what the proposer demanded. Such a multidimensional response format intends to cast further light on the motives guiding responder behavior. Using a conservative and stringent approach to type classification, we find that the overwhelming majority of responder participants choose consistently with outcomebased preference models. There are, however, few responders that destroy the proposer's demand of a large pie share and concurrently reject their own offer, thereby suggesting a strong concern for integrity.
		\end{itemize}
\end{itemize}
\vspace{-0.5cm}
\section{Cooling Off in Negotiations: Does It Work?}

\begin{itemize}
	\item Oechssler, J.; Roider, A.; Schmitz, P. (2015): Cooling Off in Negotiations: Does it Work?. Journal of Institutional and Theoretical Economics JITE J Inst Theor Econ 171, (2015). (Ultimatum Game)
		\begin{itemize}
			\item Negotiations frequently end in conflict after one party rejects a final offer. In a large-scale Internet experiment, we investigate whether a 24-hour cooling-off period leads to fewer rejections in ultimatum bargaining. We conduct a standard cash treatment and a lottery treatment, where subjects receive lottery tickets for several large prizes. In the lottery treatment, unfair offers are less frequently rejected, and cooling off reduces the rejection rate further. In the cash treatment, rejections are more frequent and remain so after cooling off. We also study the effect of subjects’ degree of “cognitive reflection” on their behaviour.
		\end{itemize}
\end{itemize}