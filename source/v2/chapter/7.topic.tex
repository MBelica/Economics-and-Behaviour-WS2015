%!TEX root = Economics and Behaviour.tex

\chapter{Organisations and Markets: The role of moral dimensions of markets}


\section{Morals and Markets}

The possibility that market interactions may erode moral values is a long-standing, but controversial hypothesis in the social sciences, ethic and philosophy. Markets are accused to transform human values in exchange blues and goods into commodities. It has also been argues that market institutions may influence preferences in general with a tendency to make people.

Michael Sandal analysed that with technological progress and the increasing ubiquity of market ideas, since markets continue to enter further and further domains of our social life.

\textit{Further, there is the doux commerce hypothesis, meaning that the entering of market in our social life might improve our situation in many ways...}

\begin{itemize}
	\item Falk, A.; Szech, N. (2013): \textit{Morals and Markets}. In: Science (moral dimensions)
		\begin{itemize}
			\item The possibility that market interaction may erode moral values is a long-standing, but controversial, hypothesis in the social sciences, ethics, and philosophy. To date, empirical evidence on decay of moral values through market interaction has been scarce. We present controlled experimental evidence on how market interaction changes how human subjects value harm and damage done to third parties. In the experiment, subjects decide between either saving the life of a mouse or receiving money. We compare individual decisions to those made in a bilateral and a multilateral market. In both markets, the willingness to kill the mouse is substantially higher than in individual decisions. Furthermore, in the multilateral market, prices for life deteriorate tremendously. In contrast, for morally neutral consumption choices, differences between institutions are small.
		\end{itemize}
\end{itemize}

Examples for market designs where the idea of introducing a free market (money based) is current:
\begin{itemize}
	\item trading markets for emission certificates. To reduce pollution by restricting emission output per country a contract was design, which nevertheless allowed trading of those certificates. M. Sandel was concerned that if we put a money value on pollution it might become less moral concerning to pollute.
	\item Allocation of organs markets: one might be able to trade an incompatible organ for an compatible if available. People started discussing if money should not be introduced in this market instead of just a trading market.
	\item Adoption: high income families might be able to provide better for adopted children and therefore could be preferred on an adoption list
	\item In California child baring is allowed to be traded for money 
\end{itemize}

Restricted markets:

\begin{itemize}
	\item Employment markets are regulated, so exploitation is not (so) present.
\end{itemize}
		
\section{You Owe Me}
\begin{itemize}
	\item Malmendier, U.; Schmidt, K. (2012): \textit{You Owe Me}. In: DOI (moral dimensions)
		\begin{itemize}
			\item In many cultures and industries gifts are given in order to influence the recipient, often at the expense of a third party. Examples include business gifts of firms and lobbyists. In a series of experiments, we show that, even without incentive or informational effects, small gifts strongly influence the recipient’s behaviour in favour of the gift giver, in particular when a third party bears the cost. Subjects are well aware that the gift is given to influence their behaviour but reciprocate nevertheless. Withholding the gift triggers a strong negative response. These findings are inconsistent with the most prominent models of social preferences. We propose an extension of existing theories to capture the observed behaviour by endogenising the “reference group” to whom social preferences are applied. We also show that disclosure and size limits are not effective in reducing the effect of gifts, consistent with our model. Financial incentives ameliorate the effect of the gift but backfire when available but not provided.
		\end{itemize}
\end{itemize}

\section{How Customers' insurance coverage induces sellers' misbehaviour}
\begin{itemize}	
	\item Kerschbamer, R.; Neururer, D.; Sutter, M. (2014): \textit{How Customers' insurance coverage induces sellers' misbehaviour in markets for credence goods}
		\begin{itemize}
			\item Markets for credence goods are characterised by informational asymmetries between expert sellers and their customers, which creates strong incentives for fraudulent behaviour of sellers that results in estimated annual costs to customers and the society as a whole of billions of dollars in the US alone. Prime examples of credence goods are all kinds of repair services, the provision of medical treatments, the sale of software programs, and the provision of taxi rides in unfamiliar cities. In this paper, we examine in a natural field experiment how insurance coverage on the side of the consumer – often prevalent on important markets such as the health care or repair services sectors – can seriously exacerbate inefficiencies in the provision of credence goods by inducing misbehaviour on the side of the seller. Specifically, we study how computer repair shops take advantage of customers’ insurance for repair costs. In a control treatment, the average repair price is about Euro 70, with the repair bill increasing to Euro 129 when the service provider is informed that the insurance would reimburse the bill. Our design allows for a decomposing of the sources of this economically impressive and statistically highly significant difference showing that this is mainly due to the over-provision of parts and overcharging of working time. Overall, our results strongly suggest that insurance coverage greatly increases the extent of misbehaviour of sellers in important sectors of the economy with potentially huge costs to customers and whole economies.
		\end{itemize}
\end{itemize}


\newpage