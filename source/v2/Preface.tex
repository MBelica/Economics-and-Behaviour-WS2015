%!TEX root = Funktionalanalysis - Vorlesung.tex


\chapter*{Preface}

This script tries to offer insight into fundamental topics in behavioural economics with regard to contents and methods and to reflect different research methods and designs of economic experiments in the field of behavioural economics. The reader will be acquainted with reading and critically evaluating current research papers in the field of behavioural economics. \\ \\

In standard economic theory an agent is assumed to be an \begriff{homo oeconomicus}, or economic man, which portrays the person as consistently rational and narrowly self-interested, usually pursuing his subjectively-defined end optimally; he therefore follows two main characteristics:
\begin{itemize}
	\item \begriff{Rationality}
	\item \begriff{Maximising his utility}
\end{itemize}
However, human behaviour often stands in contrast to this theoretical concept. Behavioural economics examines actual economic behaviour, including widespread cognitive biases and other irrationalities. Some psychological, social, cognitive and emotional factors frequently elicit this deviation from standard predictions and therefore it is commonly observed in experiments, in which economists collect data to estimate effect size, test the validity of economic theories and illuminate market mechanisms. \\
For these experiments one generally adhere to the following methodological guidelines:
	\begin{itemize}
		\item Incentivise subjects with real monetary payoffs (trustworthy).
		\item Publish full experimental instructions (transparency).
		\item Do not use deception (honesty).
		\item Avoid introducing specific, concrete context (generalisation).
	\end{itemize}
	
Throughout this script we have to keep in mind that every result can be interpreted in several reasonable way. One could see game theory as a whole as a predictive tool for the behaviour of human beings, but also as simply a suggestion for how people ought to behave. Therefore we subsequently distinguish between
\begin{itemize}
	\item \begriff{prescriptive} - means containing an indication of approval or disapproval
	\item \begriff{normative} - means relating to a given model
\end{itemize}

