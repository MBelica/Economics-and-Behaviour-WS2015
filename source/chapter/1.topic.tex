%!TEX root = Economics and Behaviour.tex


\chapter{Introduction to behavioural and experimental economics}

First we have to clarify the kind of analysation we use for a game theoretical situations we distinguish between
\begin{itemize}
	\item \begriff{prescriptive} - means containing an indication of approval or disapproval
	\item \begriff{normative} - means relating to a given model
\end{itemize}

or the kind of equilibrium

\begin{tabular}{|l|l|r|}
	\hline\hline
  			& {\textbf{complete information}} & {\textbf{incomplete information}} \\
                                                    \hline
   \textbf{static games} & Nash-Equilibrium & Bayesian-Nash-Equilibrium\arrayrulewidth2pt \\
                                               \cline{1-3}
   \textbf{dynamic games} & Perfect Nash-Equilibrium & Perfect Bayesian-Nash-Equilibrium \\ \hline\hline
\end{tabular}


todo % todo at 1. topic


\newpage