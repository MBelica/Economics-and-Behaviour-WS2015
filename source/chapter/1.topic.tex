%!TEX root = Economics and Behaviour.tex


\chapter{Introduction to behavioural and experimental economics}

The games studied in game theory are well-defined mathematical objects. To be fully defined a game must specify the following elements $\{ N, S, u \}$ where

	\begin{enumerate}
		\item $N$ is the number of players and for player $n$ would that mean $n \in \{ 1, \dotsc, N \}$.
		\item For each player we have a set of \textit{pure} strategies $S$.
		\item For each player $n \in \{1, \dotsc, N \}$ we have an expected utility function $u : S \rightarrow \MdR$
	\end{enumerate}


\section{Homo oeconomicus and the representation of games} \index{Homo oeconomicus}
First, analysing a game one ofter determines the optimal strategy for a playing by assuming that he is an homo oeconomicus. What does this imply? We assume for a homo oeconomicus two main characteristics:
\begin{enumerate}
	\item Rationality
	\item Maximising his/her utility
\end{enumerate}


This assumptions can lead to different kinds of equilibrium where the most commons are \\ \\
\begin{tabular}{|l|l|r|}
	\hline\hline
  			& {\textbf{complete information}} & {\textbf{incomplete information}} \\
                                                    \hline
   \textbf{static games} & Nash-Equilibrium & Bayesian-Nash-Equilibrium\arrayrulewidth2pt \\
                                               \cline{1-3}
   \textbf{dynamic games} & Perfect Nash-Equilibrium & Perfect Bayesian-Nash-Equilibrium \\ \hline\hline
\end{tabular}

While the following definitions hold:

\begin{definition}[static game] \index{static game}
	A static game is one in which all players make decisions (or select a strategy) simultaneously, without knowledge of the strategies that are being chosen by other players. Even though the decisions may be made at different points in time, the game is simultaneous because each player has no information about the decisions of others; thus, it is as if the decisions are made simultaneously. Simultaneous games are represented by the normal form and solved using the concept of a Nash equilibrium.
\end{definition}

\begin{definition}[dynamic game] \index{dynamic game}
When players interact by playing a similar stage game (such as the prisoner's dilemma) numerous times, the game is called a dynamic, or repeated game. Unlike simultaneous games, players have at least some information about the strategies chosen on others and thus may contingent their play on past moves.
\end{definition}

\begin{definition}[complete information]
	In a game of \begriff{complete information}, the structure of the game and the payoff functions of the players are commonly known but players may not see all of the moves made by other players (for instance, the initial placement of ships in Battleship); there may also be a chance element (as in most card games). Conversely, in games of perfect information, every player observes other players' moves, but may lack some information on others' payoffs, or on the structure of the game.
\end{definition}

\begin{definition}[incomplete information] 
A game of \begriff{incomplete information} is a game where the players do not have common knowledge of the game being played. Among the aspects of the game that the players might not have common knowledge of are:
	\begin{itemize}
		\item Payoffs
		\item Who the other players are
		\item What moves are possible
		\item How outcome depends on the action.
		\item What opponent knows, and what he knows I know....
	\end{itemize}
\end{definition}

Second, we have to clarify how we are going to interprete our results. One could see game theory as a predictive tool for the behaviour of human beings, but also as simply a suggestion for how people ought to behave. Therefore we distinguish between
\begin{itemize}
	\item \begriff{prescriptive} - means containing an indication of approval or disapproval
	\item \begriff{normative} - means relating to a given model
\end{itemize}
	

\section{Experimental economics} \index{Experimental economics}
To analyse a real-life situation with these methods one establishes a game representing the situation and collects data to estimate effect size, test the validity of economic theories, and illuminate market mechanisms. \\
For these experiments economists generally adhere to the following methodological guidelines:
	\begin{itemize}
		\item Incentivise subjects with real monetary payoffs (trustworthy).
		\item Publish full experimental instructions (transparency).
		\item Do not use deception (honesty).
		\item Avoid introducing specific, concrete context (generalisation).
	\end{itemize}


\newpage