%!TEX root = Economics and Behaviour.tex


\chapter{Introduction to behavioural and experimental economics}

First we have to clarify the kind of analysation we use for a game theoretical situations we distinguish between
\begin{itemize}
	\item \begriff{prescriptive} - means containing an indication of approval or disapproval
	\item \begriff{normative} - means relating to a given model
\end{itemize}

or the kind of equilibrium

\begin{tabular}{|l|l|r|}
	\hline\hline
  			& {\textbf{complete information}} & {\textbf{incomplete information}} \\
                                                    \hline
   \textbf{static games} & Nash-Equilibrium & Bayesian-Nash-Equilibrium\arrayrulewidth2pt \\
                                               \cline{1-3}
   \textbf{dynamic games} & Perfect Nash-Equilibrium & Perfect Bayesian-Nash-Equilibrium \\ \hline\hline
\end{tabular}
	
~\newline	
	
Where the following definitions hold (for the definition of a strategic game see chapter 2):

\begin{definition}[static game] \index{static game}
	A static game is one in which all players make decisions (or select a strategy) simultaneously, without knowledge of the strategies that are being chosen by other players. Even though the decisions may be made at different points in time, the game is simultaneous because each player has no information about the decisions of others; thus, it is as if the decisions are made simultaneously. Simultaneous games are represented by the normal form and solved using the concept of a Nash equilibrium.
\end{definition}

\begin{definition}[dynamic game] \index{dynamic game}
When players interact by playing a similar stage game (such as the prisoner's dilemma) numerous times, the game is called a dynamic, or repeated game. Unlike simultaneous games, players have at least some information about the strategies chosen on others and thus may contingent their play on past moves.
\end{definition}

\begin{definition}[complete information]
	In a game of \begriff{complete information}, the structure of the game and the payoff functions of the players are commonly known but players may not see all of the moves made by other players (for instance, the initial placement of ships in Battleship); there may also be a chance element (as in most card games). Conversely, in games of perfect information, every player observes other players' moves, but may lack some information on others' payoffs, or on the structure of the game.
\end{definition}

\begin{remark} [imperfect vs incomplete information] 
In a game of imperfect information, players are simply unaware of the actions chosen by other
players. However they know who the other players are hat their possible strategies/actions are and are, what their possible strategies/actions are, and the preferences/payoffs of these other players. Hence, information about the other players in imperfect information is complete.

In \begriff{incomplete information} games, players may or may not know some information about the other players, e.g. their “type”, their strategies, payoffs or their preferences.	
\end{remark}

\begin{prop} \index{Experimental economics}
Experimental economists generally adhere to the following methodological guidelines:
	\begin{itemize}
		\item Incentivise subjects with real monetary payoffs (trustworthy).
		\item Publish full experimental instructions (transparency).
		\item Do not use deception (honesty).
		\item Avoid introducing specific, concrete context (generalisation).
	\end{itemize}
\end{prop}


\newpage