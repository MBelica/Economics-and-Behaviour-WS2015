%!TEX root = Economics and Behaviour.tex

\section{Pleasures of Skill and Moral Conduct}

Background:
\begin{itemize}
	\item Jeremy Bentham points out fourteen different $"$simple$"$ sources of pleasures for humans
	\item In this short list, number three are the $"$pleasure of skill$"$ while number five are $"$the pleasure of a good name$"$.
	\item Yet if being skilful is of crucial importance to people then this can oppose the possibility to keep a good name
\end{itemize}
  
  
  
\textbf{As an example: The Manhattan Project.} After the dropping of the plutonium bomb on Nagasaki, numerous members of the Manhattan Project started worrying about moral implications. Many of the scientists suffered from e.g depressions. \\ 

The Self-Image is so relevant in this concept. Both the desire for mastery and acting in accordance with moral values originate from the same source, a desire for positive self-image. \\ 

The remaining question is therefore: does morality in some (everyday) situations get traded off against skilfulness?

\newpage