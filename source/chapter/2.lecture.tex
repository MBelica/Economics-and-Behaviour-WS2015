%!TEX root = Economics and Behaviour.tex

\begin{example}[Guessing-Game / Beauty-Contest] \index{Guessing-Game}
	In a game with at least two players we can describe the sequel for the Guessing-Game as following
	\begin{itemize}
		\item $n \geq 2$ players
		\item Every player guesses a number $b_{i} \in \{0, 1, 2, \dotsc, 100 \}$
		\item Goal is to guess $b_{i}$ as close as possible to $\frac{p}{n} \cdot \sum_{i = 1}^{n} b_{i} = p \cdot \varnothing, \quad p \in (0, 1)$
		\item The best guess (closes to $p \cdot \varnothing$) wins, in case of a tie a random device that is 'fair' decides who win the price $P > 0$
	\end{itemize}
	
	
	\textbf{1. Question:} Is $(0, \dotsc, 0)$ a Nash-Equilibrium? \\
	\textbf{Answer:} Yes. Assume all bidders except for bidder $i$ bid $0$.	
		\begin{itemize}
			\item if bidder $i$ bids $0$ expected win equals $\frac{1}{n} P $
			\item if bidder $i$ bids something above $0$ his expected profit is going to be $0$ as $0$ is closer to $p \cdot \varnothing$ then the bet $b > 0$ of player $i$:
			\[ p \cdot \varnothing = p \frac{(n - 1)0 + 1 b}{n} = p \frac{b}{n} \]
			\[
				d \left( b, p \frac{b}{n} \right) > d \left( p \frac{b}{n}, 0 \right)	\gdw \left| b - p \frac{b}{n} \right| > \left| p \frac{b}{n} \right| \quad
			\]
			
			and since $b > p \cdot \varnothing$ we can simplify this further
			\[ b > \left( p \cdot b \right) \cdot \frac{2}{n} \quad \text{is true because of } n \geq 2 \text{ and } p < 1.\]
		\end{itemize}
		
	
	\textbf{2. Question:} is $(0, \dotsc, 0)$ the unique Nash-Equilibrium here? \\
	\textbf{Answer:} Yes, since:	
	\[ b_{i}^{*} \leq \frac{1}{2} \frac{\sum_{j \neq i} b_{j}^{*}}{n - 1} \]
	\begin{align*}
		\Rightarrow \sum_{i = 1}^{n} b_{i}^{*} \leq \frac{1}{2} \frac{\sum_{i = 1}^{n} \sum_{j \neq i} b_{j}^{*}}{n - 1} & = \frac{1}{2} \frac{(n - 1) \sum_{j = 1}^{n} b_{j}^{*}}{n - 1} \\
		& = \frac{1}{2} \sum_{j = 1}^{n} b_{j}^{*}
	\end{align*} 
	\[ \gdw \sum_{i = 1}^{n} b_{i}^{*} \leq \frac{1}{2} \sum_{i = 1}^{n} b_{i}^{*} \]
	therefore $(0, \dotsc,  0)$ is the only Nash-Equilibrium here. \\


	\textbf{3. Question:} is $(0, \dotsc, 0)$ also a strictly dominant strategy? \\
	\textbf{Answer:} No. By analyzing the following situation we find a counterexample: \\
	$50$ players. $48$ of them bid the number $100$, bidder $49$ bids $0$ then the optimal strategy for player $50$ is to bid $97$. \\
	Therefore $0$ is not the best answer and cannot be a strictly dominant strategy.
\end{example}

\begin{example}
	something is missing in my notes here %todo missing in my notes
\end{example}

\newpage