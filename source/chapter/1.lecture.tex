%!TEX root = Economics and Behaviour.tex
\chapter{Gametheorie}
\section*{Introduction}
In analysing game theoretical situations we distinguish between the judging
\begin{itemize}
	\item \begriff{prescriptive} - means containing an indication of approval or disapproval
	\item \begriff{normative} - means relating to our model
\end{itemize}

or the kind of equilibrium

\begin{tabular}{|l|l|r|}
	\hline\hline
  			& {\textbf{complete information}} & {\textbf{incomplete information}} \\
                                                    \hline
   \textbf{static games} & Nash-Equilibrium & Bayesian-Nash-Equilibrium\arrayrulewidth2pt \\
                                               \cline{1-3}
   \textbf{dynamic games} & Perfect Nash-Equilibrium & Perfect Bayesian-Nash-Equilibrium \\ \hline\hline
\end{tabular}

\section{Games in strategic form}
A game in strategic form is completely characterised by $\{ N, S, u \}$ where
	\begin{enumerate}
		\item $N$ is the number of players and for player $n$ would that mean $n \in \{ 1, \dotsc, N \}$.
		\item For each player we have a set of \textit{pure} strategies $S$.
		\item For each player $n \in \{1, \dotsc, N \}$ we have an expected utility function $u : S \rightarrow \MdR$
	\end{enumerate}

For two players ($P1$ and $P2$) and two possible signals ($a$ and $b$) we could use the matrix form, where the $u_{i}(x, y)$ is the utility function for Player $i$ for signal $x$ for $P1$ and $y$ for $P2$ with $x, y \in \{ a, b\}$
\begin{center}
	\begin{tabular}{|c|c|c|}
		\hline\hline
  			$P1$ / $P2$ & \textbf{a} & \textbf{b} \\
         		\cline{2-3}
   					\textbf{a} & $( u_{1}(a, a) , u_{2}(a, a))$ & $(u_{1}(a, b), u_{2}(a, b))$	\arrayrulewidth2pt \\
            	\cline{2-3}
   					\textbf{b} & $( u_{1}(b, a), u_{2}(b, a))$ & $(u_{1}(b, b), u_{2}(b, b))$\\ \hline\hline
	\end{tabular}	
\end{center}

We call a \begriff{set of strategies} in a game a complete plan of actions for each situation in the game.

\begin{example}[Prisoner's Dilemma] \label{prisonersdilemma}
Two members of a criminal gang are arrested and imprisoned. Each prisoner is in solitary confinement with no means of communicating with the other. The prosecutors lack sufficient evidence to convict the pair on the principal charge. They hope to get both sentenced to a year in prison on a lesser charge. Simultaneously, the prosecutors offer each prisoner a bargain. Each prisoner is given the opportunity either to: betray the other by testifying that the other committed the crime, or to cooperate with the other by remaining silent. The offer is:
	\begin{itemize}
		\item If A and B each betray the other, each of them serves 6 years in prison
		\item If A betrays B but B remains silent, A will be set free and B will serve 9 years in prison (and vice versa)
		\item If A and B both remain silent, both of them will only serve 1 year in prison (on the lesser charge)
	\end{itemize}
	
\begin{center}
	\begin{tabular}{|l|l|r|}
		\hline\hline
  			P1 / P2 & \textbf{defects} & \textbf{cooperates} \\
         		\cline{1-3}
   			\textbf{defects} & $(-6, -6)$ & $(0, -9)$ 	\arrayrulewidth2pt \\
            	\cline{1-3}
   			\textbf{cooperates} & $(-9, 0)$ & $(-1, -1)$ \\ \hline\hline
	\end{tabular}	
\end{center}


	Other Interpretations for the Prisoner's Dilemma
	\begin{itemize}
		\item Collusion on prices
		\item Investing in human capital vs. arming for a war
		\item Buying a SUV vs. a smaller car
	\end{itemize}
\end{example}

\begin{definition*}
	A strategy $s_{i}''$ is strictly dominated if and only if there exists another strategy $s_{i}'$ such that
	\[ u(s_{i}', s_{-i}) \geq u(s_{i}'', s_{-i}) \quad \forall s_{-i} \in S_{i} \]	
	In the \hyperref[prisonersdilemma]{Prisoner's Dilemma} is $cooperate$ strictly dominated by $defect$. Simply the elimination of strictly dominated strategies leads to the prediction of $(defects, defects)$.
\end{definition*}

\begin{example}
	Iterated elimination of strictly dominated strategies leads to
	\begin{itemize}
		\item for Player 2: $l$ strictly dominates $r$
		\item after having eliminated $r$ we can further eliminate $d$, since $d$ is then strictly dominated by $u$
	\end{itemize}
\end{example}

Important to notice is that here, the prediction we derived relies immensely on the rationality of all players.

\begin{definition}[A strategy profile] \label{strategyprofile}
	We call a vector $S = (S_{1}, \dotsc, S_{N})$ of dimension $N$ that specifies a strategy for every player in the game a \begriff{strategy profile}.
\end{definition}

\begin{definition}[Nash-Equilibrium] \label{nashequilibrium}
	A informal definition for a \begriff{Nash-Equilibrium} would be that it is the mutual best response for every player, therefore a strategy profile in which no player can do better by unilaterally changing their strategy. \\
	Defining it formally would mean: a strategy profile $x^{*} \in S$ is a Nash-Equilibrium if no unilateral deviation in strategy by any single player is profitable for tat player, that is
	\[ \forall i \in \{1, \dotsc, N \}, x_{i} \in S_{i}: \quad u_{i}(x_{i}^{*}, x_{-i}^{*}) \geq u_{i}(x_{i}, x_{-i}^{*}) \]
	\end{definition}

\begin{example}[Battle of the sexes] \label{battleofthesexes}
		The next example is a two-player coordination game. Image a couple that agreed to meet this evening, but both individually cannot recall if they will be attending the opera or a football match. The husband would most of all like to go to the football game. The wife would like to go to the opera. Both would prefer to go to the same place rather than different ones. \\
		Hence this game in strategic form could look something like:
		
		\begin{center}
			\begin{tabular}{|l|l|r|}
				\hline\hline
  					M / F & \textbf{football} & \textbf{opera} \\
         				\cline{1-3}
   					\textbf{football} & $(1, 2)$ & $(0, 0)$ 	\arrayrulewidth2pt \\
            			\cline{1-3}
   					\textbf{opera} & $(0, 0)$ & $(2, 1)$ \\ \hline\hline
			\end{tabular}	
		\end{center}
		
		The two Nash-Equilibriums in this game are $(opera, opera)$ and $(football, football)$ since
		\[ u_{i}(opera, opera) \geq u_{i}(football, opera) \quad \forall i \in \{ 1, 2 \} \]
\end{example}

\begin{example}[The Beauty-Contest]
	missing in my notes. %todo: missing in my notes
\end{example}

\subsection{Presentations}
todo %todo: have to sum those presentations and but them here at some point

\newpage