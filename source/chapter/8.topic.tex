%!TEX root = Economics and Behaviour.tex


\chapter{Ethics in science}

\section{Pleasures of Skill and Moral Conduct}

Background:
\begin{itemize}
	\item Jeremy Bentham points out fourteen different $"$simple$"$ sources of pleasures for humans
	\item In this short list, number three are the $"$pleasure of skill$"$ while number five are $"$the pleasure of a good name$"$.
	\item Yet if being skilful is of crucial importance to people then this can oppose the possibility to keep a good name
\end{itemize}
  
  
  
\textbf{As an example: The Manhattan Project.} After the dropping of the plutonium bomb on Nagasaki, numerous members of the Manhattan Project started worrying about moral implications. Many of the scientists suffered from e.g depressions. \\ 

The Self-Image is so relevant in this concept. Both the desire for mastery and acting in accordance with moral values originate from the same source, a desire for positive self-image. \\ 

The remaining question is therefore: does morality in some (everyday) situations get traded off against skilfulness?

\section{Moral and Markets}


Examples for market designs where the idea of introducing (money) market/ free market is current:
\begin{itemize}
	\item trading markets for emission certificates. To reduce pollution by restricting emission output per country a contract was design, but it allowed trad. M. Sandel was concerned that if we put a money value on pollution it might be less moral concerning to pollute.
	\item Allocation of organs market, you might be able to trade an incompatible organ for an compatible if available. People start discussing if money should not be introduced in this market instead of just the trading market.
	\item Adoption: high income families can provide well for children and therefore might be preferred on the 
	\item In California child baring is allowed to be traded for money 
\end{itemize}

Restricted markets:

\begin{itemize}
	\item Employment markets are regulated, so exploitation is not so present.
\end{itemize}



The paper presented in the lecture was on the topic of "moral and markets" by N. Szech from ne year 2013/2016, it is available online but this is the topic:


Te possibility that market interactions may erode moral values is a long-standing, but controversial, hypothesis in the social sciences, ethic and philosophy. Markets are accused to transform human values in exchange blues and goods into commodities. It has also been argues that market institutions may influence preferences in general with a tendency to make people

Michael Sandal analyses that with technological progress and the increasing ubiquity of market ideas, markets continues to enter further and further domains of our social life.



But there is the doux commerce hypothesis, meaning that markets might improve us in many ways


\section{Presented papers}

\begin{itemize}
	\item Falk, A.; Szech, N. (2016): \textit{Pleasures of Skill and Moral Conduct}. KIT working paper. (non-monetary incentives and morals)
	\item Russell, B. (1960): \textit{Science}, New Series.
	\item N.N. (1961): \textit{The Moral UnNeutrality of Science: The scientist's special responsibility are examined an address given at the 1960 AAAS annual meeting}. In: Science.
\end{itemize}


\newpage