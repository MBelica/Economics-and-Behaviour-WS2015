%!TEX root = Economics and Behaviour.tex

\chapter{Organisations and Markets: The role of market incentives}


\section{Pay Enough or Don't Pay at All}
\begin{itemize}
	\item Gneezy, U.; Rustichini, A. (2000): \textit{Pay Enough or Don't Pay at All}. In: Quarterly Journal of Economics.
		\begin{itemize}
			\item Economists usually assume that monetary incentives improve performance, and psychologists claim that the opposite may happen. We present and discuss a set of experiments designed to test these contrasting claims. \\
				We found that the effect of monetary compensation on performance was not monotonic. In the treatments in which money was offered, a larger amount yielded a higher performance. However, offering money did not always produce an improvement: subjects who were offered monetary incentives performed more poorly than those who were offered no compensation. Several possible interpretations of the results are discussed.
		\end{itemize}
\end{itemize}

\section{A Fine is a Prise}
\begin{itemize}
	\item Gneezy, U.; Rustichini, A. (2000): \textit{A Fine is a Prise}. In: The Journal of Legal Studies. (monetary incentives)
		\begin{itemize}
			\item The deterrence hypothesis predicts that the introduction of a penalty that leaves everything else unchanged will reduce the occurrence of the behaviour subject to the fine. We present the result of a field study in a group of day-care centers that contra- dicts this prediction. Parents used to arrive late to collect their children, forcing a teacher to stay after closing time. We introduced a monetary fine for late-coming parents. As a result, the number of late-coming parents increased significantly. After the fine was removed no reduction occurred. We argue that penalties are usually introduced into an incomplete contract, social or private. They may change the information that agents have, and therefore the effect on behaviour may be opposite of that expected. If this is true, the deterrence hypothesis loses its predictive strength, since the clause ‘‘everything else is left unchanged’’ might be hard to satisfy.
		\end{itemize}
\end{itemize}

\section{The Currency or Reciprocity}
\begin{itemize}
	\item Sebastian Kube, Michel Andre Marechal and Clemens Puppe (2012): \textit{The Currency or Reciprocity: Gift Exchange in the Workplace}. In: American Economics Review. (money versus non-monetary incentives)
		\begin{itemize}
			\item The psychological impact of providing tangible or intangible gifts to employees is likely to depend not only on the magnitude of the gifts but also on the gifts being seen as (...) costly to the donor in terms of time or effort.
		\end{itemize}
\end{itemize}

\section{Creativity and Financial Incentives}
\begin{itemize}
	\item Charness, G.; Grieco, D. (2014): \textit{Creativity and Financial Incentives} 
		\begin{itemize}
			\item Creativity is a complex and multi-dimensional phenomenon with tremendous economics importance. Yet, despite this importance, there is very little work on the topic in the economics literature. In this paper, we consider the effect of incentives on creativity. We present a first series of experiments on individual creativity where subjects face creativity tasks where, in one case, ex-ante goals and constraints are imposed on their answers, and in the other case no restrictions apply. The effects of financial incentives in stimulating creativity in both types of tasks is then tested, together with the impact of personal features like risk and ambiguity aversion. Our findings show that, in general, financial incentives affect “closed” (constrained) creativity, but do not facilitate “open” (unconstrained) creativity. However, in the latter case incentives do play a role for ambiguity-averse agents, who tend to be significantly less creative and seem to need extrinsic motivation to exert effort in a task whose odds of success they don’t know. The second set of experiments aims at exploring group creativity in contexts where the “corporate culture” is either cooperative or individualistic. Our results show that, in the case of closed tasks, financial incentives and collectivist attitudes foster creativity, but only with cooperative corporate culture.
		\end{itemize}
\end{itemize}


\newpage