\chapter*{Exams} \addcontentsline{toc}{chapter}{Exams}

Here, the memory minutes of the past exam from 19. February 2016:

\textbf{Exercise 1} (19 Points)
\begin{enumerate}[label=\alph*\upshape)]
	\item Define the term \textit{homo oeconomicus} (4 Points)
	\item Sketch the structure of an ultimatum game and provide a game tree. (8 Points)
	\item Explain the design of the experiment in Oechssler, J., Roider A., Schmitz, P. (2015): \textit{Cooling Off in Negotiations: Does It Work?}. (7 Points)
\end{enumerate}
\textbf{Exercise 2} (21 Points)
\begin{enumerate}[label=\alph*\upshape)]
	\item Illustrate the design used for the experiment in the paper Charness, G., Grieco, D.; (2014): \textit{Creativity and Financial Incentives}. (8 Points)
	\item Elaborate one aspect you liked and one aspect you disliked about the design. (4 + 4 Points)
	\item Suggest one reasonable extension to the design. (5 Points)
\end{enumerate}
\textbf{Exercise 3} (20 Points)
In a modified Beauty-Contest with 5 players each player $i$ has to choose a number $x_{i}$ between 0 and 100, winner is the one closest to half times the average ($\frac{1}{2} \sum_{i = 1}^{n} \frac{x_{i}}{5}$). If a draw occurs the player with the smallest index $i$ wins the toss.
\begin{enumerate}[label=\alph*\upshape)]
	\item Define level-0, level-1, level-2 and level-3. (5 Points)
	\item Is $(0, \dotsc, 0)$ a Nash-Equilibrium? (5 Points)
	\item Is $0$ a dominant strategy? (5 Points)
	\item Is $(0, \dotsc, 0)$ the unique Nash-Equilibrium? (5 Points)
\end{enumerate}