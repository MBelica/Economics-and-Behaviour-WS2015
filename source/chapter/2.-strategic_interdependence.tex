%!TEX root = Economics and Behaviour.tex

\chapter{How strategic interdependence can change subjective decisions}

\section{An ultimatum game with multidimensional response strategies} 

\begin{itemize}
	\item Güth, W.; Levati, M.V.; Nardi, C.; Soraperra, I. (2014): \textit{An ultimatum game with multidimensional response strategies}. In Jena Economic Research Papers, FriedrichSchiller University and Max Planck Institute of Economics, Jena, Germany (Ultimatum Game)
		\begin{itemize}
			\item We enrich the choice task of responders in ultimatum games by allowing them to independently decide whether to collect what is offered to them and whether to destroy what the proposer demanded. Such a multidimensional response format intends to cast further light on the motives guiding responder behaviour. Using a conservative and stringent approach to type classification, we find that the overwhelming majority of responder participants choose consistently with outcome based preference models. There are, however, few responders that destroy the proposer's demand of a large pie share and concurrently reject their own offer, thereby suggesting a strong concern for integrity.
		\end{itemize}
\end{itemize}
\vspace{-0.5cm}
\section{Cooling Off in Negotiations: Does It Work?}

\begin{itemize}
	\item Oechssler, J.; Roider, A.; Schmitz, P. (2015): Cooling Off in Negotiations: Does it Work?. Journal of Institutional and Theoretical Economics JITE J Inst Theor Econ 171, (2015). (Ultimatum Game)
		\begin{itemize}
			\item Negotiations frequently end in conflict after one party rejects a final offer. In a large-scale Internet experiment, we investigate whether a 24-hour cooling-off period leads to fewer rejections in ultimatum bargaining. We conduct a standard cash treatment and a lottery treatment, where subjects receive lottery tickets for several large prizes. In the lottery treatment, unfair offers are less frequently rejected, and cooling off reduces the rejection rate further. In the cash treatment, rejections are more frequent and remain so after cooling off. We also study the effect of subjects’ degree of “cognitive reflection” on their behaviour.
		\end{itemize}
\end{itemize}

\section{Level k as a prominent example of a nonstandard/behavioural approach}


\textbf{Unraveling in Guessing Games: An Experimental Study}

\begin{itemize}
	\item Nagel, R. (1995): \textit{Unraveling in Guessing Games: An Experimental Study}. In: American Economic Review.
		\begin{itemize}
			\item Consider the following game: a large number of players have to state in several rounds simultaneously a number in the closed interval [0, 100]. The winner is the person whose chosen number is closest to the mean of all chosen numbers multiplied by a parameter $p$, where $p$ is common knowledge. The payoff to the winner is a fixed amount, which is independent of the stated number and $p$. If there is a tie, the prize is divided equally among the winners. The other players whose chosen numbers are further away receive nothing.
		\end{itemize}
\end{itemize}	
			
\textbf{More than Meets the Eye}

\begin{itemize}
	\item Müller, J.; Schwieren, C. (2011): \textit{More than Meets the Eye: an Eye-tracking Experiment on the Beauty Contest Game}
		\begin{itemize}
			\item The beauty contest game has been used to analyse how many steps of reasoning subjects are able to perform. A common finding is that a majority seem to have low levels of reasoning. We use eye-tracking to investigate not only the number chosen in the game, but also the strategies in use and the numbers contemplated. We can show that not all cases that are seemingly level-1 or level-2 thinking indeed are – they might be highly sophisticated adaptations to beliefs about other people’s limited reasoning abilities.
		\end{itemize}
\end{itemize}		

\newpage