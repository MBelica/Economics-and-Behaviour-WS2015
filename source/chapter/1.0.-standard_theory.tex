%!TEX root = Economics and Behaviour.tex


\chapter{Standard theoretic basics for analysis of strategic behaviour}

First, a strategic interaction occurs when the utility of agents in a situation is mutually influenced by individual behavioural changes. Any time we have a situation with two or more agents that involves known payouts or quantifiable consequences, we generally use games to help determine the most likely outcomes. 

	A \begriff{game} is a formal representation of a situation in which a number of individuals interact in a setting of strategic interaction. 

	With this in mind, it is necessary to clarify a few terms commonly used in the study of games:
	\begin{itemize}
		\item The players: A set of strategic agents making decision within the context of the game, i.e. who is interacting?	
		\item The rules: A mode of conduct recognised as binding, e.g. when do the players move or what can they do?
		\item The outcomes: For each possible set of actions by the players there is a resulting state of the game.
		\item The payoffs: Arriving at a particular outcome a player receives a certain payout (it can be in any quantifiable form, from dollars to utility). What are the players' preferences over the possible outcomes? 
	\end{itemize}

Examples are Tick-Tack-Toe games, auctions or even meetings which all can be represented by games. \\

In the following two sections, the equilibrium is going to be one of our main focus point as it one likely outcome of a game. Such a state, where economic forces are balanced, and, in the absence of external influences, the behaviour leading to the equilibrium will not change, can be described by three properties: \index{equilibrium}
\begin{enumerate}
	\item The behaviour of agents in such a point is consistent.
	\item No agent has an incentive to change his behaviour.
	\item The equilibrium is the outcome of some dynamic process (stability).
\end{enumerate}

~\newline
Furthermore, we subsequently study different formal representations which we use to model the conflict situation. However, all situations we will examine will have complete information, i.e. perfect knowledge (esp. structure of the game, possible actions and payoff functions of all players) is available to all individuals. 


\newpage