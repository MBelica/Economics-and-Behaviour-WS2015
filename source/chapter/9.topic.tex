%!TEX root = Economics and Behaviour.tex

\chapter{Non-standard utility}


\section{Anticipatory utility}

The standard utility approach states an already deterministic situation on an individuals anticipatable behaviour and this means his utility function is static and can't be changed by additional information. Nevertheless:

\begin{itemize}
	\item Some students decide not to look up their exam grades while on vacation, therefore they refuse gathering free and more important static information to (better) enjoy their free time
	\item Some people with potentially severe diseases avoid getting tested for them.
\end{itemize}

One could argue that even with a bad result they don't have to act upon it, they don't have to behave differently, so why do this situation occur? 

Maybe learning about the future affects well-being today derived from their \textbf{beliefs} about the future.

In Psychology one distinguishes between:
\begin{itemize}
	\item monitors: people who really want to know what is going to happen. E.g. some people want to know every step of their upcoming surgery even though it won't change the outcome
	\item blunders: subjects who don't want the additional information
\end{itemize}

\textbf{Behavioural Economics} by Caplin/Leahy (2001, 2004) tries to combine those two fields

Maybe some people prefer to stick to their Bayesian's priors instead of getting tested because they incorporate their \textbf{beliefs} into their well-being (utility)

~\newline

What if there is an instrumental cost in getting tested?
\begin{itemize}
	\item Caplin/Eliaz (2003): examined social cost (e.g. HIV tests in america)
	\item Köszegi (2003, 2006): (studied the some problem as the next paper)
	\item Szech/Schweizer (2015): look at individual well-being as instrumental cost
\end{itemize}

As solution is proposed in both papers the one from Caplin/Eliaz and the one from Szech/Schweizer: coarse tests may be helpful.

~\newline

Some people even bias their own beliefs away from the Bayesian:
 
Brunnermeier and Parker (2005) and also Oster, Shoulsen, Dorsey (2013) showed that some people might have high risk of inheriting diseases but can convince themselves that the risk is way lower, where this is more than simple optimism.

\section{Presented papers}

\begin{itemize}
	\item Stefano DellaVigna and Ulrike Malmendier : \textit{Paying Not to Go to the Gym}.
	\item George Loewenstein: \textit{Because It Is There: The Challenge of Mountaineering... for Utility Theory}.
	\item Ananda Ganguly and Joshua Tasoff (2014): \textit{Fantasy and Dread: An Experimental of Attentional Anticipatory Utility}
\end{itemize}


\newpage