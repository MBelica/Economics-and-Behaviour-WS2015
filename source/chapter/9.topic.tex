%!TEX root = Economics and Behaviour.tex

\chapter{Non-standard utility}


\section{Anticipatory utility}

The standard utility approach states an already deterministic situation on an individuals anticipatable behaviour and this means his utility function is static and can't be changed by additional information. Nevertheless:

\begin{itemize}
	\item Some students decide not to look up their exam grades while on vacation, therefore they refuse gathering free and more important static information to (better) enjoy their free time
	\item Some people with potentially severe diseases avoid getting tested for them.
\end{itemize}

One could argue that even with a bad result they don't have to act upon it, they don't have to behave differently, so why do this situation occur? 

Maybe learning about the future affects well-being today derived from their \textbf{beliefs} about the future.

In Psychology one distinguishes between:
\begin{itemize}
	\item monitors: people who really want to know what is going to happen. E.g. some people want to know every step of their upcoming surgery even though it won't change the outcome
	\item blunders: subjects who don't want the additional information
\end{itemize}

\textbf{Behavioural Economics} by Caplin/Leahy (2001, 2004) tries to combine those two fields

Maybe some people prefer to stick to their Bayesian's priors instead of getting tested because they incorporate their \textbf{beliefs} into their well-being (utility)
~\newline

What if there is an instrumental cost in getting tested?
\begin{itemize}
	\item Caplin/Eliaz (2003): examined social cost (e.g. HIV tests in america)
	\item Köszegi (2003, 2006): (studied the some problem as the next paper)
	\item Szech/Schweizer (2015): look at individual well-being as instrumental cost
\end{itemize}

As solution is proposed in both papers the one from Caplin/Eliaz and the one from Szech/Schweizer: coarse tests may be helpful.
~\newline

Some people even bias their own beliefs away from the Bayesian:
 
Brunnermeier and Parker (2005) and also Oster, Shoulsen, Dorsey (2013) showed that some people might have high risk of inheriting diseases but can convince themselves that the risk is way lower, where this is more than simple optimism.

\section{Presented papers}

\begin{itemize}
	\item Stefano DellaVigna and Ulrike Malmendier : \textit{Paying Not to Go to the Gym}. (26 pages)
		\begin{itemize}
			\item How do consumers choose from a menu of contracts? We analyse a novel dataset from three U.S. health clubs with information on both the contractual choice and the day-to-day attendance decisions of 7,752 members over three years. The observed consumer behaviour is difficult to reconcile with standard preferences and beliefs. First, members who choose a contract with a flat monthly fee of over \$70 attend on average 4.3 times per month. They pay a price per expected visit of more than \$17, even though they could pay \$10 per visit using a 10-visit pass. On average, these users forgo savings of \$600 during their membership. Second, consumers who choose a monthly contract are 17 percent more likely to stay enrolled beyond one year than users committing for a year. This is surprising because monthly members pay higher fees for the option to cancel each month. We also document cancellation delays and attendance expectations, among other findings. Leading explanations for our findings are overconfidence about future self-control or about future efficiency. Overconfident agents overestimate attendance as well as the cancellation probability of automatically renewed contracts. Our results suggest that making inferences from observed contract choice under the rational expectation hypothesis can lead to biases in the estimation of consumer preferences.
		\end{itemize}
	\item George Loewenstein: \textit{Because It Is There: The Challenge of Mountaineering... for Utility Theory}. (15 pages)
		\begin{itemize}
			\item This paper presents experimental evidence for an intrinsic preference for information. In two experiments we find that the demand for information about a future experience, controlling for its usefulness, is increasing in the expected future consumption utility. In the first experiment subjects obtain information about the outcome of a lottery now or later. The information is useless for decision making, but the larger the reward, the more likely subjects are to pay to obtain the information early. In the second experiment subjects may pay to avoid being tested for herpes simplex virus 1 and the more highly feared herpes simplex virus 2. Subjects are about twice more likely to avoid information for herpes simplex virus 2, suggesting that more aversive outcomes lead to more information avoidance. In addition, we find that positive affect (i.e. good mood) is associated with lower demand for information as predicted by theory, and information avoidance is associated with ambiguity aversion.
		\end{itemize}
	\item Ananda Ganguly and Joshua Tasoff (2014): \textit{Fantasy and Dread: An Experimental of Attentional Anticipatory Utility}. (66 pages)
		\begin{itemize}
			\item This paper presents experimental evidence for an intrinsic preference for information. In two experiments we find that the demand for information about a future experience, controlling for its usefulness, is increasing in the expected future consumption utility. In the first experiment subjects obtain information about the outcome of a lottery now or later. The information is useless for decision making, but the larger the reward, the more likely subjects are to pay to obtain the information early. In the second experiment subjects may pay to avoid being tested for herpes simplex virus 1 and the more highly feared herpes simplex virus 2. Subjects are about twice more likely to avoid information for herpes simplex virus 2, suggesting that more aversive outcomes lead to more information avoidance. In addition, we find that positive affect (i.e. good mood) is associated with lower demand for information as predicted by theory, and information avoidance is associated with ambiguity aversion. \\ \\
				
				Critic: just a small fraction was actually being tested. 
		\end{itemize}
\end{itemize}


\newpage