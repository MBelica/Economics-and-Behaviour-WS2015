%!TEX root = Economics and Behaviour.tex

\chapter{Non-standard utility}


\section{Anticipatory utility}

The standard utility approach would state an already deterministic situation and it's utility for an idividuum is not changed by additional informations. Nevertheless:

\begin{itemize}
	\item Some students may decide not to look up their exam grades while on vacation, therefore they refuse gathering free informations to e.g. not disturb the free time
	\item Also people with potentially severe diseases avoid getting tested for them
\end{itemize}

One could argue that even with a bad result they don't have to act upon it, so why do this situation occur? 

Maybe learning about the future affects well-being today derived from their \textbf{beliefs} about the future.

In Psychology one distingueshes between:
\begin{itemize}
	\item monitors: people who really want to know what is going to happen. E.g. some people want to know every step of their upcoming surgery even though it won't change the outcome
	\item blunters: subjects who don't want the additional informations
\end{itemize}


\textbf{Behavioral Economics} by Caplin/Leahy (2001, 2004) try to combine those two sections

Maybe some people prefer to stick to their Bayesians priors instead of getting tested because they incorporate their \textbf{beliefs} into their well-being (utility)


What if there is an instrumental cost in getting tested?
\begin{itemize}
	\item Caplin/Eliaz (2003): social cost (e.g. HIV tests in america)
	\item Köszegi (2003, 2006): (same as next)
	\item Szech/Schweizer (2015) : individual well-being
\end{itemize}

Both papers the one from Caplin/Eliaz and the one from Szech/Schweizer they come to the conclusion that coarse tests may be helpful.

Brunnermeier and Parker (2005) and also Oster, Shoulsen, Dorsey (2013) showed that some people might have high risk of inheriting diseases but can convince themselves that the risk is way lower, where this is more then simple optimism. \\
Maybe sometimes people are able to bias their beliefs away from the Bayesian. 


\newpage