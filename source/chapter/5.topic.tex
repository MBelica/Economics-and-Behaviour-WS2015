%!TEX root = Economics and Behaviour.tex

\chapter{Level k as a prominent example of a nonstandard/behavioural approach}


\section{Presented papers}

\begin{itemize}
	\item Nagel, R. (1995): \textit{Unraveling in Guessing Games: An Experimental Study}. In: American Economic Review.
		\begin{itemize}
			\item \textbf{Unraveling in Guessing Games: An Experimental Study} \\ 
			Consider the following game: a large number of players have to state in several rounds simultaneously a number in the closed interval [0, 100]. The winner is the person whose chosen number is closest to the mean of all chosen numbers multiplied by a parameter $p$, where $p$ is common knowledge. The payoff to the winner is a fixed amount, which is independent of the stated number and $p$. If there is a tie, the prize is divided equally among the winners. The other players whose chosen numbers are further away receive nothing.
			
			The game is played with four groups each having a different parameter $p$ and each group is playing four rounds, all communication between participants was denied. After every round the chosen numbers, the mean and the winning number was read aloud.
			
			We can categories players be their depth of calculation, in the simplest case, a player selects a strategy at random without forming beliefs or pricks a number that is salient to him and call this zero-order belief. Analogous we define first-order belief, the player thinks that others select a number at random and he chooses his best response to this belief. But we assume that there is only a finit depth of reasoning for all players and call the highest the n-order beliefs
			
			There have been already similar studies, e.g. by John M. Keynes who described the mental process of competitor confronted with picking the face in a newspaper that is closest to the mean preference of all competitors.
			
			Judith Mehta studied behaviour in two-person coordination games, suggesting that players coordinate by either applying depth of reasoning of order 1 or by picking a focal point.
			
			Comparing both of these papers to the given study reveals that those were one-shot games. In this experiment, the decisions in first period indicate that depth of reasoning of order 1 and 2 may be playing a significant role. In periods 2-4 for $p < 1$, one find that the modal depth of reasoning does not increase, although the median choice decreases over time.
			
			Therefore a simple qualitative learning theory based on individual experience is proposed as a better explanation of behaviour over time than a modal of increasing depth of reasoning.
			
			One can argue that a person is strategic of degree n if he chooses the number $50p^{n}$. For proving this the author used only nonparametric tests. 
	
		\end{itemize}
	\item Müller, J.; Schwieren, C. (2011): \textit{More than Meets the Eye: an Eye-tracking Experiment on the Beauty Contest Game}
		\begin{itemize}
			\item \textbf{More than Meets the Eye: an Eye-tracking Experiment on the Beauty Contest Game}  \\
			The beauty contest game has been used to analyse how many steps of reasoning subjects are able to perform. A common finding is that a majority seem to have low levels of reasoning. We use eye-tracking to investigate not only the number chosen in the game, but also the strategies in use and the numbers contemplated. We can show that not all cases that are seemingly level-1 or level-2 thinking indeed are – they might be highly sophisticated adaptations to beliefs about other people’s limited reasoning abilities.
		\end{itemize}
\end{itemize}		

\newpage